% arara: lualatex
% arara: clean: { extensions: [ acn,acr,alg,aux,bbl,bcf,blg,glg,glo,gls,idx,ilg,ind,ist,log,lof,lol,lot,out,ptc,toc,run.xml ]}

%!TEX TS-program = xelatex
%!TEX encoding = UTF-8 Unicode
% Awesome CV LaTeX Template for Cover Letter
%
% This template has been downloaded from:
% https://github.com/posquit0/Awesome-CV
%
% Authors:
% Claud D. Park <posquit0.bj@gmail.com>
% Lars Richter <mail@ayeks.de>
%
% Template license:
% CC BY-SA 4.0 (https://creativecommons.org/licenses/by-sa/4.0/)
%


%-------------------------------------------------------------------------------
% CONFIGURATIONS
%-------------------------------------------------------------------------------
% A4 paper size by default, use 'letterpaper' for US letter
\documentclass[11pt, a4paper]{awesome-cv}

% Configure page margins with geometry
\geometry{left=1.4cm, top=.8cm, right=1.4cm, bottom=1.8cm, footskip=.5cm}

% Color for highlights
% Awesome Colors: awesome-emerald, awesome-skyblue, awesome-red, awesome-pink, awesome-orange
%                 awesome-nephritis, awesome-concrete, awesome-darknight
\colorlet{awesome}{awesome-orange}
% Uncomment if you would like to specify your own color
% \definecolor{awesome}{HTML}{CA63A8}

% Colors for text
% Uncomment if you would like to specify your own color
% \definecolor{darktext}{HTML}{414141}
% \definecolor{text}{HTML}{333333}
% \definecolor{graytext}{HTML}{5D5D5D}
% \definecolor{lighttext}{HTML}{999999}
% \definecolor{sectiondivider}{HTML}{5D5D5D}

% Set false if you don't want to highlight section with awesome color
\setbool{acvSectionColorHighlight}{true}

% If you would like to change the social information separator from a pipe (|) to something else
\renewcommand{\acvHeaderSocialSep}{\quad\textbar\quad}


%-------------------------------------------------------------------------------
%	PERSONAL INFORMATION
%	Comment any of the lines below if they are not required
%-------------------------------------------------------------------------------
% Available options: circle|rectangle,edge/noedge,left/right
%\photo[circle,noedge,left]{./profile}
\name{}{LAI Kwun Hang (Adrian)}
\position{Phd Candidate}
%\address{235, World Cup buk-ro, Mapo-gu, Seoul, 03936, Republic of Korea}

\mobile{(+31) 6-2184-1392}
\email{adrian.k.h.lai@gmail.com}
%\dateofbirth{January 1st, 1970}
%\homepage{www.posquit0.com}
\github{Adriankhl}
\linkedin{adrian-k-h-lai}
\orcid{0000-0003-0446-119X}
% \gitlab{gitlab-id}
% \stackoverflow{SO-id}{SO-name}
% \twitter{@twit}
% \skype{skype-id}
% \reddit{reddit-id}
% \medium{madium-id}
% \kaggle{kaggle-id}
% \googlescholar{googlescholar-id}{name-to-display}
%% \firstname and \lastname will be used
% \googlescholar{googlescholar-id}{}
% \extrainfo{extra information}

\quote{``We are all in the gutter, but some of us are looking at the stars"}


%-------------------------------------------------------------------------------
%	LETTER INFORMATION
%	All of the below lines must be filled out
%-------------------------------------------------------------------------------
% The company being applied to
\recipient
  {Adyen}
  {Amsterdam}
% The date on the letter, default is the date of compilation
\letterdate{\today}
% The title of the letter
\lettertitle{Job Application for Java Software Engineer - Data}
% How the letter is opened
\letteropening{Dear Hiring Manager,}
% How the letter is closed
\letterclosing{Sincerely,}
% Any enclosures with the letter
%\letterenclosure[Attached]{Curriculum Vitae}


%-------------------------------------------------------------------------------
\begin{document}

% Print the header with above personal information
% Give optional argument to change alignment(C: center, L: left, R: right)
\makecvheader[R]

% Print the footer with 3 arguments(<left>, <center>, <right>)
% Leave any of these blank if they are not needed
\makecvfooter
  {}%\today}
  {}%Claud D. Park~~~·~~~Cover Letter}
  {}

% Print the title with above letter information
\makelettertitle

%-------------------------------------------------------------------------------
%	LETTER CONTENT
%-------------------------------------------------------------------------------
\begin{cvletter}

\lettersection{About Me}
I identify myself as a physicist, in a social science department, working mostly on software development.
During my master's research in astrophysics,
I learned to work with Linux, supercomputers, simulation, and data analysis.
Then, I came to Leiden to do my PhD in social science,
which I gained some experiences in databases, textual analysis, machine learning, network analysis,
and agent-based simulation.

Because of some catastrophic engineering mistakes made by the developers in a foreign institute,
my primary research project became impossible to continue.
To rescue my PhD, I decided to dive deep into software development.
I dedicated over 2 years to develop an open source agent-based simulation framework for interstellar social model
in 4D, relativistic spacetime.
On top of the framework, I have also developed a turn-based strategy game with server-client architecture
and GUI that works on PC and Android.
I am quite proud of being able to leverage my knowledge of physics, social science, and software development
to independently create such a cool software from scratch, and I have shared it in multiple social science conferences
and software development conferences.

I am also an enthusiastic open source contributor. 
Whenever I come across a bug in an open source software,
I always try to fix it and contribute back a pull request if possible.
I particularly enjoy working on C++ projects because it helps me
to build up my engineering skill for lower-level details,
which I rarely had the chance to practice in my social science research projects.

\lettersection{Why the Netherlands eScience Center?}
The failure of my primary PhD research project taught me a lesson -
it is of utmost importance to have competitive research software engineers.
I have talked to several people from the eScience Center last year,
and I am impressed by the idea of gathering software talents in a place to support research.
So far, the center seems to be a pretty successful social experiment.
As a technical person who enjoys intellectual challenges,
I am excited to see the variety of software projects within the center.

In conclusion, my quantitative background, my experience in social science research,
and my capability of doing proper software development make me a good fit
for the eScience Center and specifically this position.
Thank you for your consideration, I am looking forward to hearing back from your team.

%\lettersection{Why Me?}

\end{cvletter}


%-------------------------------------------------------------------------------
% Print the signature and enclosures with above letter information
\makeletterclosing

\end{document}
