% arara: lualatex
% arara: clean: { extensions: [ acn,acr,alg,aux,bbl,bcf,blg,glg,glo,gls,idx,ilg,ind,ist,log,lof,lol,lot,out,ptc,toc,run.xml ]}

%!TEX TS-program = xelatex
%!TEX encoding = UTF-8 Unicode
% Awesome CV LaTeX Template for Cover Letter
%
% This template has been downloaded from:
% https://github.com/posquit0/Awesome-CV
%
% Authors:
% Claud D. Park <posquit0.bj@gmail.com>
% Lars Richter <mail@ayeks.de>
%
% Template license:
% CC BY-SA 4.0 (https://creativecommons.org/licenses/by-sa/4.0/)
%


%-------------------------------------------------------------------------------
% CONFIGURATIONS
%-------------------------------------------------------------------------------
% A4 paper size by default, use 'letterpaper' for US letter
\documentclass[11pt, a4paper]{awesome-cv}

% Configure page margins with geometry
\geometry{left=1.4cm, top=.8cm, right=1.4cm, bottom=1.8cm, footskip=.5cm}

% Color for highlights
% Awesome Colors: awesome-emerald, awesome-skyblue, awesome-red, awesome-pink, awesome-orange
%                 awesome-nephritis, awesome-concrete, awesome-darknight
\colorlet{awesome}{awesome-orange}
% Uncomment if you would like to specify your own color
% \definecolor{awesome}{HTML}{CA63A8}

% Colors for text
% Uncomment if you would like to specify your own color
% \definecolor{darktext}{HTML}{414141}
% \definecolor{text}{HTML}{333333}
% \definecolor{graytext}{HTML}{5D5D5D}
% \definecolor{lighttext}{HTML}{999999}
% \definecolor{sectiondivider}{HTML}{5D5D5D}

% Set false if you don't want to highlight section with awesome color
\setbool{acvSectionColorHighlight}{true}

% If you would like to change the social information separator from a pipe (|) to something else
\renewcommand{\acvHeaderSocialSep}{\quad\textbar\quad}


%-------------------------------------------------------------------------------
%	PERSONAL INFORMATION
%	Comment any of the lines below if they are not required
%-------------------------------------------------------------------------------
% Available options: circle|rectangle,edge/noedge,left/right
%\photo[circle,noedge,left]{./profile}
\name{}{LAI Kwun Hang (Adrian)}
\position{Phd Candidate}
%\address{235, World Cup buk-ro, Mapo-gu, Seoul, 03936, Republic of Korea}

\mobile{(+31) 6-2184-1392}
\email{adrian.k.h.lai@gmail.com}
%\dateofbirth{January 1st, 1970}
%\homepage{www.posquit0.com}
\github{Adriankhl}
\linkedin{adrian-k-h-lai}
\orcid{0000-0003-0446-119X}
% \gitlab{gitlab-id}
% \stackoverflow{SO-id}{SO-name}
% \twitter{@twit}
% \skype{skype-id}
% \reddit{reddit-id}
% \medium{madium-id}
% \kaggle{kaggle-id}
% \googlescholar{googlescholar-id}{name-to-display}
%% \firstname and \lastname will be used
% \googlescholar{googlescholar-id}{}
% \extrainfo{extra information}

\quote{``We are all in the gutter, but some of us are looking at the stars"}


%-------------------------------------------------------------------------------
%	LETTER INFORMATION
%	All of the below lines must be filled out
%-------------------------------------------------------------------------------
% The company being applied to
\recipient
  {Adyen}
  {Amsterdam}
% The date on the letter, default is the date of compilation
\letterdate{\today}
% The title of the letter
\lettertitle{Job Application for Java Software Engineer - Data}
% How the letter is opened
\letteropening{Dear Hiring Manager,}
% How the letter is closed
\letterclosing{Sincerely,}
% Any enclosures with the letter
%\letterenclosure[Attached]{Curriculum Vitae}


%-------------------------------------------------------------------------------
\begin{document}

% Print the header with above personal information
% Give optional argument to change alignment(C: center, L: left, R: right)
\makecvheader[R]

% Print the footer with 3 arguments(<left>, <center>, <right>)
% Leave any of these blank if they are not needed
\makecvfooter
  {}%\today}
  {}%Claud D. Park~~~·~~~Cover Letter}
  {}

% Print the title with above letter information
\makelettertitle

%-------------------------------------------------------------------------------
%	LETTER CONTENT
%-------------------------------------------------------------------------------
\begin{cvletter}

\lettersection{About Me}
I identify myself as a physicist, in a social science department, working mostly on software development.
Because of my interdisciplinary research background,
I have experience in processing, analyzing and visualizing a variety of data in natural science and social science.
The typical data pipeline in my current institute involves queries from a Microsoft SQL Server,
which teaches me solid knowledge in SQL.

Over the last two and a half years,
I have been working on a Kotlin project for my PhD -
an agent-based simulation framework and a turn-based strategy game in 4D, relativistic spacetime,
such that people can utilize it to study interstellar societies.
The game has a http-based server-client architecture for multiplayer and a functioning GUI,
and the computation and communication are parallelized by coroutine.
The software follows good software development practice, e.g., having unit testing and logging.
I am quite proud of being able to create such a cool software from scratch,
and I have presented the project in many different conferences including KotlinConf.
Since Kotlin is closely related to Java,
I have the confident to be a competent Java developer as well.

Besides academic work, I am an enthusiastic open source contributor. 
Whenever I come across a bug in an open source software,
I always try to fix it and contribute back a pull request if possible.
It is fun to learn about new projects and pick up new programming languages.

\lettersection{Why Adyen?}
As someone from academia,
I am always curious about how it is like to work on real-world application
on a global scale.
Adyen, being a top player in the payment industry,
provides such an opportunity.
Being in the Netherlands for a while,
I really appreciate Dutch's inclusiveness and directness,
which are also the core values of Adyen.
The competitiveness and the vision of Adyen make it an attractive workplace.

In conclusion, my technical capabilities and data-oriented research experiences
fit well for this position.
Thank you for your consideration, I am looking forward to hearing back from your team.

%\lettersection{Why Me?}

\end{cvletter}


%-------------------------------------------------------------------------------
% Print the signature and enclosures with above letter information
\makeletterclosing

\end{document}

\iffalse
Please, describe the project you are most proud of. What was the technical problem and how did you solve it? What was your contribution to the project?

I am very proud of my project - Relativitization, which is an agent-based simulation framework and also a turn-based strategy game in 4D, relativistic spacetime. Social scientists can use the framework to build models to study interstellar society, and others can just have fun with the game. The software is written in Kotlin. I am the creator and the sole developer of the project. You can find it at https://github.com/Adriankhl/relativitization

The framework is required to be performant, physically correct, and easy to use. Special relativity tells us that there is time delay and time dilation, which affects what an agent can see and how they can interact with each other. Therefore, I have to craft the data structure, algorithm, and API carefully to fulfill the need. I have also parallelized the computation using coroutine. It is handy to have immutable data to ensure the correctness of the parallelized computation, so I have created a small code-generation library KSergen (https://github.com/Adriankhl/ksergen) to generate immutable data classes from mutable data classes.

The game must have a GUI, and I would like it to have multiplayer functionality. Since I was not a professional game developer, I had to figure them out myself. I implemented the server-client architecture with the Ktor library, where messages are transferred through the HTTP protocol. I built the GUI based on libGDX, which works on both PC and Android devices.

In general, it is quite challenging to turn a cool idea into a software with more than 60k lines of code, while keeping everything to function well together. I surely have become a more mature developer after the project.
\fi

