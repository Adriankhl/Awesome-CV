% arara: lualatex
% arara: clean: { extensions: [ acn,acr,alg,aux,bbl,bcf,blg,glg,glo,gls,idx,ilg,ind,ist,log,lof,lol,lot,out,ptc,toc,run.xml ]}

%!TEX TS-program = xelatex
%!TEX encoding = UTF-8 Unicode
% Awesome CV LaTeX Template for Cover Letter
%
% This template has been downloaded from:
% https://github.com/posquit0/Awesome-CV
%
% Authors:
% Claud D. Park <posquit0.bj@gmail.com>
% Lars Richter <mail@ayeks.de>
%
% Template license:
% CC BY-SA 4.0 (https://creativecommons.org/licenses/by-sa/4.0/)
%


%-------------------------------------------------------------------------------
% CONFIGURATIONS
%-------------------------------------------------------------------------------
% A4 paper size by default, use 'letterpaper' for US letter
\documentclass[11pt, a4paper]{awesome-cv}

% Configure page margins with geometry
\geometry{left=1.4cm, top=.8cm, right=1.4cm, bottom=1.8cm, footskip=.5cm}

% Color for highlights
% Awesome Colors: awesome-emerald, awesome-skyblue, awesome-red, awesome-pink, awesome-orange
%                 awesome-nephritis, awesome-concrete, awesome-darknight
\colorlet{awesome}{awesome-orange}
% Uncomment if you would like to specify your own color
% \definecolor{awesome}{HTML}{CA63A8}

% Colors for text
% Uncomment if you would like to specify your own color
% \definecolor{darktext}{HTML}{414141}
% \definecolor{text}{HTML}{333333}
% \definecolor{graytext}{HTML}{5D5D5D}
% \definecolor{lighttext}{HTML}{999999}
% \definecolor{sectiondivider}{HTML}{5D5D5D}

% Set false if you don't want to highlight section with awesome color
\setbool{acvSectionColorHighlight}{true}

% If you would like to change the social information separator from a pipe (|) to something else
\renewcommand{\acvHeaderSocialSep}{\quad\textbar\quad}


%-------------------------------------------------------------------------------
%	PERSONAL INFORMATION
%	Comment any of the lines below if they are not required
%-------------------------------------------------------------------------------
% Available options: circle|rectangle,edge/noedge,left/right
%\photo[circle,noedge,left]{./profile}
\name{}{LAI Kwun Hang (Adrian)}
\position{Phd Candidate}
%\address{235, World Cup buk-ro, Mapo-gu, Seoul, 03936, Republic of Korea}

\mobile{(+31) 6-2184-1392}
\email{adrian.k.h.lai@gmail.com}
%\dateofbirth{January 1st, 1970}
%\homepage{www.posquit0.com}
\github{Adriankhl}
\linkedin{adrian-k-h-lai}
\orcid{0000-0003-0446-119X}
% \gitlab{gitlab-id}
% \stackoverflow{SO-id}{SO-name}
% \twitter{@twit}
% \skype{skype-id}
% \reddit{reddit-id}
% \medium{madium-id}
% \kaggle{kaggle-id}
% \googlescholar{googlescholar-id}{name-to-display}
%% \firstname and \lastname will be used
% \googlescholar{googlescholar-id}{}
% \extrainfo{extra information}

\quote{``We are all in the gutter, but some of us are looking at the stars"}


%-------------------------------------------------------------------------------
%	LETTER INFORMATION
%	All of the below lines must be filled out
%-------------------------------------------------------------------------------
% The company being applied to
\recipient
  {JetBrains}
  {Amsterdam, Netherlands}
% The date on the letter, default is the date of compilation
\letterdate{\today}
% The title of the letter
\lettertitle{Job Application for QA Engineer (Kotlin/JVM team)}
% How the letter is opened
\letteropening{Dear Hiring Manager,}
% How the letter is closed
\letterclosing{Sincerely,}
% Any enclosures with the letter
%\letterenclosure[Attached]{Curriculum Vitae}


%-------------------------------------------------------------------------------
\begin{document}

% Print the header with above personal information
% Give optional argument to change alignment(C: center, L: left, R: right)
\makecvheader[R]

% Print the footer with 3 arguments(<left>, <center>, <right>)
% Leave any of these blank if they are not needed
\makecvfooter
  {}%\today}
  {}%Claud D. Park~~~·~~~Cover Letter}
  {}

% Print the title with above letter information
\makelettertitle

%-------------------------------------------------------------------------------
%	LETTER CONTENT
%-------------------------------------------------------------------------------
\begin{cvletter}

I am writing to express my enthusiastic interest in the "QA Engineer (Kotlin/JVM team)" position at JetBrains.

As a dedicated Kotlin indie game developer, the prospect of working at JetBrains and contributing to the evolution of the Kotlin language itself truly excites me. The role perfectly aligns with my passion for quality assurance and my familiarity with the Kotlin ecosystem.

My journey started in the realm of physics, honing my skills in C, Python, Linux systems, supercomputing, and simulations. These technical proficiencies equipped me to tackle intricate challenges. My PhD research, a fusion of computational social science, expanded my horizons. Amid my doctoral pursuits, I ventured into crafting a sophisticated simulation framework and a game using Kotlin. My joy in developing with Kotlin is evident, showcasing my deep understanding of the language, its ecosystem, unit/integration testing, and its interoperability with Java.

In parallel with my academic endeavors, I actively contributed to open-source projects, transforming me from a scientific programmer to a well-rounded software engineer. A standout example is my KSerGen library, born after I discovered KSP at KotlinConf 2023. This library automates the generation of immutable data classes and serialization hierarchies, reducing boilerplate code in my Kotlin games. These experiences underscore my commitment to software excellence and my adaptability in the face of new challenges.

I am eager to bring my technical capabilities and passion for Kotlin to the QA Engineer role at JetBrains. Thank you for considering my application. I am eagerly looking forward to the opportunity to discuss how my experiences align with the requirements of the role in greater detail.

%\lettersection{Why Me?}

\end{cvletter}


%-------------------------------------------------------------------------------
% Print the signature and enclosures with above letter information
\makeletterclosing

\end{document}
