% arara: xelatex
% arara: biber
% arara: xelatex
% arara: clean: { extensions: [ acn,acr,alg,aux,bbl,bcf,blg,glg,glo,gls,idx,ilg,ind,ist,log,lof,lol,lot,out,ptc,toc,run.xml ]}


%!TEX TS-program = xelatex
%!TEX encoding = UTF-8 Unicode
% Awesome CV LaTeX Template for CV/Resume
%
% This template has been downloaded from:
% https://github.com/posquit0/Awesome-CV
%
% Author:
% Claud D. Park <posquit0.bj@gmail.com>
% http://www.posquit0.com
%
% Template license:
% CC BY-SA 4.0 (https://creativecommons.org/licenses/by-sa/4.0/)
%


%-------------------------------------------------------------------------------
% CONFIGURATIONS
%-------------------------------------------------------------------------------
% A4 paper size by default, use 'letterpaper' for US letter
\documentclass[11pt, a4paper]{awesome-cv}

% Configure page margins with geometry
\geometry{left=1.4cm, top=.8cm, right=1.4cm, bottom=1.8cm, footskip=.5cm}

% Specify the location of the included fonts
\fontdir[fonts/]

% Color for highlights
% Awesome Colors: awesome-emerald, awesome-skyblue, awesome-red, awesome-pink, awesome-orange
%                 awesome-nephritis, awesome-concrete, awesome-darknight
\colorlet{awesome}{awesome-red}
% Uncomment if you would like to specify your own color
% \definecolor{awesome}{HTML}{CA63A8}

% Colors for text
% Uncomment if you would like to specify your own color
% \definecolor{darktext}{HTML}{414141}
% \definecolor{text}{HTML}{333333}
% \definecolor{graytext}{HTML}{5D5D5D}
% \definecolor{lighttext}{HTML}{999999}
% \definecolor{sectiondivider}{HTML}{5D5D5D}

% Set false if you don't want to highlight section with awesome color
\setbool{acvSectionColorHighlight}{true}

% If you would like to change the social information separator from a pipe (|) to something else
\renewcommand{\acvHeaderSocialSep}{\quad\textbar\quad}


%-------------------------------------------------------------------------------
%	PERSONAL INFORMATION
%	Comment any of the lines below if they are not required
%-------------------------------------------------------------------------------
% Available options: circle|rectangle,edge/noedge,left/right
% \photo{./examples/profile.png}
\name{}{LAI Kwun Hang (Adrian)}
\position{Phd Candidate}
%\address{42-8, Bangbae-ro 15-gil, Seocho-gu, Seoul, 00681, Rep. of KOREA}

\mobile{(+31) 6-2184-1392}
\email{k.h.lai@cwts.leidenuniv.nl}
%\dateofbirth{January 1st, 1970}
%\homepage{www.posquit0.com}
\github{Adriankhl}
%\linkedin{posquit0}
% \gitlab{gitlab-id}
% \stackoverflow{SO-id}{SO-name}
% \twitter{@twit}
% \skype{skype-id}
% \reddit{reddit-id}
% \medium{medium-id}
% \kaggle{kaggle-id}
% \googlescholar{googlescholar-id}{name-to-display}
%% \firstname and \lastname will be used
% \googlescholar{googlescholar-id}{}
% \extrainfo{extra information}
\orcid{0000-0003-0446-119X}

%\quote{``Be the change that you want to see in the world."}

%-------------------------------------------------------------------------------
%	BIBLIOGRAPHY
%-------------------------------------------------------------------------------
\addbibresource{cv/my-representative-publication.bib}
\addbibresource{cv/ligo-scientific-collaboration.bib}


%-------------------------------------------------------------------------------
\begin{document}

% Print the header with above personal information
% Give optional argument to change alignment(C: center, L: left, R: right)
\makecvheader

% Print the footer with 3 arguments(<left>, <center>, <right>)
% Leave any of these blank if they are not needed
\makecvfooter
  {\today}
  {LAI Kwun Hang~~~·~~~Curriculum Vitae}
  {\thepage}


%-------------------------------------------------------------------------------
%	CV/RESUME CONTENT
%	Each section is imported separately, open each file in turn to modify content
%-------------------------------------------------------------------------------
%-------------------------------------------------------------------------------
%	SECTION TITLE
%-------------------------------------------------------------------------------
\cvsection{Education}


%-------------------------------------------------------------------------------
%	CONTENT
%-------------------------------------------------------------------------------
\begin{cventries}

%---------------------------------------------------------
  \cventry
    {Ph.D. in Quantitative Science Studies} % Degree
    {Centre for Science and Technology Studies, Leiden University} % Institution
    {Leiden, The Netherlands} % Location
    {2019 - present (Expected graduation: Early 2024)} % Date(s)
    {}

%---------------------------------------------------------
  \cventry
    {M.Phil. in Physics} % Degree
    {The Chinese University of Hong Kong} % Institution
    {Hong Kong, China} % Location
    {2016 - 2018} % Date(s)
    {}

%---------------------------------------------------------
  \cventry
    {B.Sc. in Physics} % Degree
    {United College, The Chinese University of Hong Kong} % Institution
    {Hong Kong, China} % Location
    {2012 - 2016} % Date(s)
    {
      \begin{cvitems} % Description(s) bullet points
        \item {Minor in Mathematics}
      \end{cvitems}
    }

%---------------------------------------------------------
\end{cventries}

%-------------------------------------------------------------------------------
%	SECTION TITLE
%-------------------------------------------------------------------------------
\cvsection{Research Highlights}


%-------------------------------------------------------------------------------
%	CONTENT
%-------------------------------------------------------------------------------
\begin{cventries}

%---------------------------------------------------------
  \cventry
    {Relativitization: an agent-based simulation framework in 4D, relativistic spacetime} % Research project
    {Research software development} % Research organization
    {Leiden, The Netherlands} % Location
    {2021 - present} % Date(s)
    {
      \begin{cvitems} % Description(s) of tasks/responsibilities
        \item {Create a computational framework for social scientists to simulate interstellar agent-based models.}
        \item {Build models to demonstrate the effect relativistic physics on interstellar societies.}
        \item {Develop a video game on top of the framework.}
        \item {Programming language: Kotlin.}
      \end{cvitems}
    }

%---------------------------------------------------------
  \cventry
    {Criteria Project} % Research project
    {Research on Research Institute (RORI)} % Research organization
    {Leiden, The Netherlands} % Location
    {2020 - present} % Date(s)
    {
      \begin{cvitems} % Description(s) of tasks/responsibilities
        \item {Construct Bayesian models on the relation between the gender bias and criteria of funding application.}
        \item {Perform data cleaning and analysis.}
        \item {Programming language: Python.}
      \end{cvitems}
    }

%---------------------------------------------------------
  \cventry
    {Gravitational waves modeling and data analysis} % Research project
    {LIGO Scientific Collaboration (LSC) Projects} % Research organization
    {Hong Kong, China} % Location
    {2016 - 2018} % Date(s)
    {
      \begin{cvitems} % Description(s) of tasks/responsibilities
        \item {Proposed methods to constrain black-hole horizon effects.}
        \item {Ran simulations of detecting intermediate-mass black hole by gravitational wave lensing.}
        \item {Programming language: C and Python.}
      \end{cvitems}
    }

%---------------------------------------------------------
%  \cventry
%    {Majorana Stellar Representation of Quantum Entanglement} % Research project
%    {Undergraduate Research Project} % Research organization
%    {Hong Kong, China} % Location
%    {2015} % Date(s)
%    {
%      \begin{cvitems} % Description(s) of tasks/responsibilities
%        \item {Visualized spin-1/2 particle states by Majorana stars.}
%      \end{cvitems}
%    }

%---------------------------------------------------------
\end{cventries}

%-------------------------------------------------------------------------------
%	SECTION TITLE
%-------------------------------------------------------------------------------
\cvsection{Selected Academic Publications}

%-------------------------------------------------------------------------------
%	SUBSECTION TITLE
%-------------------------------------------------------------------------------
\cvsubsection{Conference Proceedings}

\begin{refsection}
    \nocite{lai2021issi}

	\printbibliography[
	heading=none, 
	sorting=ydnt
	]
\end{refsection}

%-------------------------------------------------------------------------------
%	SUBSECTION TITLE
%-------------------------------------------------------------------------------
\cvsubsection{Journal Articles}

\begin{refsection}
	\nocite{lai2018constraining}
	\nocite{lai2018discovering}

	\printbibliography[
	heading=none, 
	sorting=ydnt
	]
\end{refsection}

%-------------------------------------------------------------------------------
%	SECTION TITLE
%-------------------------------------------------------------------------------
\cvsection{Presentations}


%-------------------------------------------------------------------------------
%	CONTENT
%-------------------------------------------------------------------------------
\begin{cventries}


%---------------------------------------------------------
  \cventry
    {Behavioural models of funding applications} % Role
    {Funded and Unfunded Science: Academic Inequalities and Epistemic Gaps} % Event
    {Prague, Czech Republic} % Location
    {2021} % Date(s)
    {
      \begin{cvitems} % Description(s)
        \item {Introduced my research on constructing simulation models of funding applicants' behaviours.}
      \end{cvitems}
    }

%---------------------------------------------------------
  \cventry
    {Constructing models of academic funding system: relation between the amount of funding and the efficiency} % Role
    {The 18th International Conference of the International Society for Scientometrics and Informetrics} % Event
    {Belgium (Virtual)} % Location
    {2021} % Date(s)
    {
      \begin{cvitems} % Description(s)
        \item {Introduced my research on utilizing game theoretical models and agent-based models to study the academic funding system.}
      \end{cvitems}
    }

%---------------------------------------------------------
  \cventry
    {Challenges in Using Bibliometric Indicators to Assess Peer Review Decisions: A Simulation Model} % Role
    {PEERE International Conference on Peer Review 2020} % Event
    {Spain (Virtual)} % Location
    {2020} % Date(s)
    {
      \begin{cvitems} % Description(s)
        \item {Illustrated a potential problem when using bibliometric indicators to assess peer reviews.}
      \end{cvitems}
    }

%---------------------------------------------------------
  \cventry
    {Discovering intermediate‑mass black hole lenses through gravitational wave lensing} % Role
    {Gravity and Cosmology 2018} % Event
    {Yukawa Institute for Theoretical Physics, Kyoto University, Japan} % Location
    {2018} % Date(s)
    {
      \begin{cvitems} % Description(s)
        \item {Introduced my research on utilizing gravitational waves lensing to search for intermediate‑mass black holes.}
      \end{cvitems}
    }

%---------------------------------------------------------
  \cventry
    {Gravitational lensing of gravitational waves} % Role
    {First CUHK Physics Student Conference 2017} % Event
    {The Chinese University of Hong Kong, China} % Location
    {2017} % Date(s)
    {
      \begin{cvitems} % Description(s)
        \item {Introduced my research on gravitational lensing of gravitational waves.}
        \item {Best speaker of the first CUHK Physics Student Conference.}
      \end{cvitems}
    }

%---------------------------------------------------------
\end{cventries}

%-------------------------------------------------------------------------------
%	SECTION TITLE
%-------------------------------------------------------------------------------
\cvsection{Skills}


%-------------------------------------------------------------------------------
%	CONTENT
%-------------------------------------------------------------------------------
\begin{cvskills}

%---------------------------------------------------------
  \cvskill
    {Programming} % Category
    {Kotlin, Java, Python, Julia, C/C++, Bash, LaTeX} % Skills

%---------------------------------------------------------
  \cvskill
    {Languages} % Category
    {Chinese (Cantonese and Mandarin), English} % Skills

%---------------------------------------------------------
\end{cvskills}

%-------------------------------------------------------------------------------
%	SECTION TITLE
%-------------------------------------------------------------------------------
\cvsection{Other Academic Publications}

%-------------------------------------------------------------------------------
%	SUBSECTION TITLE
%-------------------------------------------------------------------------------
\cvsubsection{LIGO-Virgo Collaboration Publications}

\begin{refsection}
    \nocite{ligo2017gravitational}
    \nocite{abbott2017all}
    \nocite{abbott2018constraints}
    \nocite{abbott2017estimating}
    \nocite{abbott2017first}
    \nocite{abbott2017firstnarrow}
    \nocite{abbott2018first}
    \nocite{abbott2018full}
    \nocite{abbott2017gravitational}
    \nocite{scientific2017gw170104}
    \nocite{abbott2017gw170608}
    \nocite{abbott2017gw170814}
    \nocite{abbott2017gw170817}
    \nocite{abbott2018gw170817}
    \nocite{abbott2018gw170817measurements}
    \nocite{abbott2017multi}
    \nocite{abbott2017progenitor}
    \nocite{abbott2019properties}
    \nocite{abbott2017searchscorpius}
    \nocite{collaboration2017search}
    \nocite{abbott2017searchintermediate}
    \nocite{abbott2017searchpost}
    \nocite{abbott2018searchsubsolar}
    \nocite{abbott2018searchtensor}
    \nocite{abbott2017upper}

	\printbibliography[
	heading=none, 
	sorting=ydnt
	]
\end{refsection}

%-------------------------------------------------------------------------------
%	SECTION TITLE
%-------------------------------------------------------------------------------
\cvsection{Honors \& Awards}


%-------------------------------------------------------------------------------
%	SUBSECTION TITLE
%-------------------------------------------------------------------------------
%\cvsubsection{International}


%-------------------------------------------------------------------------------
%	CONTENT
%-------------------------------------------------------------------------------
\begin{cvhonors}

%---------------------------------------------------------
  \cvhonor
    {Lee Kam Woon Prize (United College)} % Award
    {Given to a student for academic excellence} % Event
    {Hong Kong} % Location
    {2016} % Date(s)

%---------------------------------------------------------
  \cvhonor
    {College Head's List} % Award
    {United College} % Event
    {Hong Kong} % Location
    {2015} % Date(s)


%---------------------------------------------------------
  \cvhonor
    {Silver Medalist} % Award
    {International Physics Olympiad} % Event
    {Estonia} % Location
    {2012} % Date(s)

%---------------------------------------------------------
\end{cvhonors}

%%-------------------------------------------------------------------------------
%	SECTION TITLE
%-------------------------------------------------------------------------------
\cvsection{Research/Work Eperience}


%-------------------------------------------------------------------------------
%	CONTENT
%-------------------------------------------------------------------------------
\begin{cventries}

%---------------------------------------------------------
  \cventry
    {Self-employment} % Research project
    {Indie game developer} % Research organization
    {Leiden, The Netherlands} % Location
    {Mar. 2021 - present} % Date(s)
    {
      \begin{cvitems} % Description(s) of tasks/responsibilities
        \item{Independently developed a turn-based strategy game in Kotlin, incorporating sophisticated relativistic physics}
        \item{Implemented a HTTP-based server-client architecture and user-friendly graphical interfaces for PC and Android}
        \item{Created utility-based AI to determine actions from complex observations}
        \item{Managed a codebase with 60K+ lines of code via Git}
        \item{Presented the game at various conferences, including FOSDEM and KotlinConf}
      \end{cvitems}
    }

%---------------------------------------------------------
  \cventry
    {Researcher} % Research project
    {Research on Research Institute (RORI)} % Research organization
    {Leiden, The Netherlands} % Location
    {Jan. 2020 - present} % Date(s)
    {
      \begin{cvitems} % Description(s) of tasks/responsibilities
        \item{Constructed Bayesian models to analyze the impact of funding application criteria on gender biases}
        \item{Developed agent-based models in Java/Python/Julia to study the behavior of funding applicants}
        \item{Conducted data analysis using Python and SQL, powered by AWS, PostgreSQL, and MSSQL}
        \item{Collaborated with an international team spanning 5 countries}
      \end{cvitems}
    }

%---------------------------------------------------------
  \cventry
    {Researcher} % Research project
    {LIGO Scientific Collaboration (LSC)} % Research organization
    {Hong Kong, China} % Location
    {Sep. 2016 - Aug. 2018} % Date(s)
    {
      \begin{cvitems} % Description(s) of tasks/responsibilities
        \item{Proposed models to detect black-hole horizon effects and gravitational wave lensing}
        \item{Contributed to a C codebase for model implementation}
        \item{Ran simulations on supercomputer clusters}
        \item{Published 2 papers in reputable journals}
      \end{cvitems}
    }

%---------------------------------------------------------
%  \cventry
%    {Majorana Stellar Representation of Quantum Entanglement} % Research project
%    {Undergraduate Research Project} % Research organization
%    {Hong Kong, China} % Location
%    {2015} % Date(s)
%    {
%      \begin{cvitems} % Description(s) of tasks/responsibilities
%        \item {Visualized spin-1/2 particle states by Majorana stars.}
%      \end{cvitems}
%    }

%---------------------------------------------------------
\end{cventries}

%%-------------------------------------------------------------------------------
%	SECTION TITLE
%-------------------------------------------------------------------------------
\cvsection{Extracurricular Activity}


%-------------------------------------------------------------------------------
%	CONTENT
%-------------------------------------------------------------------------------
\begin{cventries}

%---------------------------------------------------------
  \cventry
    {Core Member} % Affiliation/role
    {B10S (B1t 0n the Security, Underground hacker team)} % Organization/group
    {S.Korea} % Location
    {Nov. 2011 - PRESENT} % Date(s)
    {
      \begin{cvitems} % Description(s) of experience/contributions/knowledge
        \item {Gained expertise in penetration testing areas, especially targeted on web application and software.}
        \item {Participated on a lot of hacking competition and won a good award.}
        \item {Held several hacking competitions non-profit, just for fun.}
      \end{cvitems}
    }

%---------------------------------------------------------
  \cventry
    {Member} % Affiliation/role
    {WiseGuys (Hacking \& Security research group)} % Organization/group
    {S.Korea} % Location
    {Jun. 2012 - PRESENT} % Date(s)
    {
      \begin{cvitems} % Description(s) of experience/contributions/knowledge
        \item {Gained expertise in hardware hacking areas from penetration testing on several devices including wireless router, smartphone, CCTV and set-top box.}
        \item {Trained wannabe hacker about hacking technique from basic to advanced and ethics for white hackers by hosting annual Hacking Camp.}
      \end{cvitems}
    }

%---------------------------------------------------------
  \cventry
    {Core Member \& President at 2013} % Affiliation/role
    {PoApper (Developers' Network of POSTECH)} % Organization/group
    {Pohang, S.Korea} % Location
    {Jun. 2010 - Jun. 2017} % Date(s)
    {
      \begin{cvitems} % Description(s) of experience/contributions/knowledge
        \item {Reformed the society focusing on software engineering and building network on and off campus.}
        \item {Proposed various marketing and network activities to raise awareness.}
      \end{cvitems}
    }

%---------------------------------------------------------
  \cventry
    {Member} % Affiliation/role
    {PLUS (Laboratory for UNIX Security in POSTECH)} % Organization/group
    {Pohang, S.Korea} % Location
    {Sep. 2010 - Oct. 2011} % Date(s)
    {
      \begin{cvitems} % Description(s) of experience/contributions/knowledge
        \item {Gained expertise in hacking \& security areas, especially about internal of operating system based on UNIX and several exploit techniques.}
        \item {Participated on several hacking competition and won a good award.}
        \item {Conducted periodic security checks on overall IT system as a member of POSTECH CERT.}
        \item {Conducted penetration testing commissioned by national agency and corporation.}
      \end{cvitems}
    }

%---------------------------------------------------------
  \cventry
    {Member} % Affiliation/role
    {MSSA (Management Strategy Club of POSTECH)} % Organization/group
    {Pohang, S.Korea} % Location
    {Sep. 2013 - Jun. 2017} % Date(s)
    {
      \begin{cvitems} % Description(s) of experience/contributions/knowledge
        \item {Gained knowledge about several business field like Management, Strategy, Financial and marketing from group study.}
        \item {Gained expertise in business strategy areas and inisght for various industry from weekly industry analysis session.}
      \end{cvitems}
    }

%---------------------------------------------------------
\end{cventries}

%%-------------------------------------------------------------------------------
%	SECTION TITLE
%-------------------------------------------------------------------------------
\cvsection{Writing}


%-------------------------------------------------------------------------------
%	CONTENT
%-------------------------------------------------------------------------------
\begin{cventries}

%---------------------------------------------------------
  \cventry
    {Founder \& Writer} % Role
    {A Guide for Developers in Start-up} % Title
    {Facebook Page} % Location
    {Jan. 2015 - PRESENT} % Date(s)
    {
      \begin{cvitems} % Description(s)
        \item {Drafted daily news for developers in Korea about IT technologies, issues about start-up.}
      \end{cvitems}
    }

%---------------------------------------------------------
  \cventry
    {Undergraduate Student Reporter} % Role
    {AhnLab} % Title
    {S.Korea} % Location
    {Oct. 2012 - Jul. 2013} % Date(s)
    {
      \begin{cvitems} % Description(s)
        \item {Drafted reports about IT trends and Security issues on AhnLab Company magazine.}
      \end{cvitems}
    }

%---------------------------------------------------------
\end{cventries}

%%-------------------------------------------------------------------------------
%	SECTION TITLE
%-------------------------------------------------------------------------------
\cvsection{Program Committees}


%-------------------------------------------------------------------------------
%	CONTENT
%-------------------------------------------------------------------------------
\begin{cvhonors}

%---------------------------------------------------------
  \cvhonor
    {Organizer} % Position
    {The 27th International Conference on Science, Technology and Innovation Indicators} % Committee
    {The Netherlands} % Location
    {2023} % Date(s)

%---------------------------------------------------------
  \cvhonor
    {Organizer} % Position
    {Workshop on the sustainability of network analysis software} % Committee
    {The Netherlands} % Location
    {2023} % Date(s)

%---------------------------------------------------------
\end{cvhonors}



%-------------------------------------------------------------------------------
\end{document}
